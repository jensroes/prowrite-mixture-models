\clearpage
\makeatletter
\efloat@restorefloats
\makeatother


\begin{appendix}
\section{}
\hypertarget{l2-effect-spl2}{%
\subsection{L2 effect (SPL2)}\label{l2-effect-spl2}}

For the SPL2 data (only the \texttt{\_\^{}{[}shift{]}}
sentence-transitions) we calculated the L2 effect (i.e.~the difference
between writing in L2 and L1). The results can be found in Table
\ref{tab:l2effect}. The results show longer keystrokes and more pauses
across all transition locations. However, the pause duration only
differed at before-word locations, not at within-word or before-sentence
transition locations.

\begin{table}[tbp]

\begin{center}
\begin{threeparttable}

\caption{\label{tab:l2effect}Mixture model estimates for language effect. Cell means are shown for transitions for writing in L1 and L2, the slowdown for long transitions and the probability of hesitant transitions. The language difference is shown on log scale (for transition durations) and logit scale for probability of hesitant transitions. 95\% PIs in brackets.}

\small{

\begin{tabular}{lrrrr}
\toprule
Transition location & \multicolumn{1}{c}{L1} & \multicolumn{1}{c}{L2} & \multicolumn{1}{c}{Difference} & \multicolumn{1}{c}{BF}\\
\midrule
Fluent transitions &  &  &  & \\
\ \ \ before sentence & 240 [216, 266] & 296 [253, 343] & 0.21 [0.08, 0.33] & 11.81\\
\ \ \ before word & 188 [173, 205] & 259 [236, 284] & 0.32 [0.28, 0.36] & > 100\\
\ \ \ within word & 138 [127, 150] & 156 [143, 169] & 0.12 [0.1, 0.14] & > 100\\
Hesitation duration &  &  &  & \\
\ \ \ before sentence & 2,469 [2,119, 2,855] & 2,769 [2,348, 3,236] & -0.08 [-0.24, 0.07] & 0.14\\
\ \ \ before word & 343 [289, 404] & 759 [667, 862] & 0.33 [0.22, 0.44] & > 100\\
\ \ \ within word & 138 [93, 196] & 171 [132, 217] & 0.05 [-0.16, 0.26] & 0.12\\
Hesitation probability &  &  &  & \\
\ \ \ before sentence & .50 [.41, .59] & .72 [.63, .80] & 0.97 [0.44, 1.5] & > 100\\
\ \ \ before word & .34 [.26, .42] & .59 [.51, .68] & 1.06 [0.58, 1.54] & > 100\\
\ \ \ within word & .07 [.05, .10] & .18 [.13, .24] & 1.08 [0.56, 1.63] & > 100\\
\bottomrule
\addlinespace
\end{tabular}

}

\begin{tablenotes}[para]
\normalsize{\textit{Note.} PIs are probability intervals. BF is the evidence in favour of the alternative hypothesis over the null hypothesis.}
\end{tablenotes}

\end{threeparttable}
\end{center}

\end{table}
\end{appendix}
