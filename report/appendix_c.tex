\clearpage
\makeatletter
\efloat@restorefloats
\makeatother


\begin{appendix}
\section{}
\hypertarget{key-combination-effect-spl2}{%
\subsection{Key-combination effect
(SPL2)}\label{key-combination-effect-spl2}}

The analysed datasets differ to the extent that keystroke intervals at
before sentence location sometimes did (PLanTra, LIFT) or did not (CATO,
C2L1, SPL2, GUNNEXP2) scope over the character following the shift key.
In other words, the pause before sentences sumed across two key
intervals in the PLanTra and LIFT data, namely
\texttt{\_\^{}{[}shift{]}\^{}C} but only involved one key interval,
namely \texttt{\_\^{}{[}shift{]}} for the remaining datasets. Therefore,
longer or more frequent pauses at before-sentence locations compared to
before-word locations can be explained without reference to linguistic
edges. Also there is a possibility that some inconsistencies in our
findings can be explained on the basis of including the keystroke
following shift.

Therefore we compared whether the different patterns can be explain on
the basis of the additional keystroked involved in before-sentence
transitions. We compared the SPL2 data including and excluding the
keystroke after shift. Although we modelled all transition locations, we
present only before-sentence transitions below as there was, as one
would expect, no difference at word locations. The results of this
comparison can be found in Table \ref{tab:shiftcellmeans}. Overall,
fluent transition duration and the hesitation duration were affected by
whether or not the sentence-initial transition include the character
following shift. Fluent key transitions were substantially longer when
including the interval following the shift key. The slowdown for
hesitations was affected too but the difference is numerically small.
There was no conclusive evidence for an increased hesitation
probability.

\begin{center}
\begin{ThreePartTable}

\begin{TableNotes}[para]
\normalsize{\textit{Note.} PIs are probability intervals. BF is the evidence in favour of the alternative hypothesis over the null hypothesis.}
\end{TableNotes}

\footnotesize{

\begin{longtable}{lrrrr}\noalign{\getlongtablewidth\global\LTcapwidth=\longtablewidth}
\caption{\label{tab:shiftcellmeans}Mixture model estimates for key transitions. Cell means are shown for transitions that do and do not involve the transition to the character following shift in msecs for fluent key-transitions, the slowdown for long transitions and the probability of hesitant transitions. The difference for including the transition duration to the character after shift is shown on log scale (for transition durations) and logit scale for probability of hesitant transitions. 95\% PIs in brackets.}\\
\toprule
Language & \multicolumn{1}{c}{\_\textasciicircum{}[shift] + C} & \multicolumn{1}{c}{\_\textasciicircum{}[shift]} & \multicolumn{1}{c}{Difference} & \multicolumn{1}{c}{BF}\\
\midrule
\endfirsthead
\caption*{\normalfont{Table \ref{tab:shiftcellmeans} continued}}\\
\toprule
Language & \multicolumn{1}{c}{\_\textasciicircum{}[shift] + C} & \multicolumn{1}{c}{\_\textasciicircum{}[shift]} & \multicolumn{1}{c}{Difference} & \multicolumn{1}{c}{BF}\\
\midrule
\endhead
Fluent transitions &  &  &  & \\
\ \ \ L1 & 390 [350, 434] & 240 [216, 266] & 0.48 [0.34, 0.63] & > 100\\
\ \ \ L2 & 448 [379, 521] & 296 [253, 343] & 0.41 [0.19, 0.63] & 46.08\\
Hesitation duration &  &  &  & \\
\ \ \ L1 & 2,398 [2,001, 2,836] & 2,469 [2,119, 2,855] & -0.46 [-0.61, -0.3] & > 100\\
\ \ \ L2 & 2,859 [2,407, 3,368] & 2,769 [2,348, 3,236] & -0.34 [-0.51, -0.17] & > 100\\
Hesitation probability &  &  &  & \\
\ \ \ L1 & .62 [.53, .71] & .50 [.41, .59] & 0.49 [-0.04, 1.03] & 1.43\\
\ \ \ L2 & .81 [.73, .88] & .72 [.63, .80] & 0.48 [-0.16, 1.13] & 0.94\\
\bottomrule
\addlinespace
\insertTableNotes
\end{longtable}

}

\end{ThreePartTable}
\end{center}
\end{appendix}
