\clearpage
\makeatletter
\efloat@restorefloats
\makeatother


\begin{appendix}
\section{}
\hypertarget{key-combination-effect-constrained-mixture-model}{%
\subsection{Key-combination effect (constrained mixture
model)}\label{key-combination-effect-constrained-mixture-model}}

The analysed datasets differ to the extent that keystroke intervals at
before sentence location sometimes did (PLanTra, LIFT) or did not (CATO,
C2L1, SPL2, GUNNEXP2) scope over the character following the shift key.
In other words, the pause before sentences summed across two key
intervals in the PLanTra and LIFT data, namely
\texttt{\_\^{}{[}shift{]}\^{}C} but only involved one key interval,
namely \texttt{\_\^{}{[}shift{]}} for the remaining datasets. Therefore,
longer or more frequent pauses at before-sentence locations compared to
before-word locations can be explained without reference to linguistic
edges.

Therefore we compared whether the different patterns can be explain on
the basis of the additional keystroke involved in before-sentence
transitions. We compared the SPL2 data including and excluding the
keystroke after shift. Although we modelled all transition locations, we
present only before-sentence transitions below as there was, as one
would expect, no difference at word locations. The results of this
comparison can be found in Table \ref{tab:shiftcellmeans}.

\begin{center}
\begin{ThreePartTable}

\begin{TableNotes}[para]
\normalsize{\textit{Note.} PIs are probability intervals. BF is the evidence in favour of the alternative hypothesis over the null hypothesis.}
\end{TableNotes}

\footnotesize{

\begin{longtable}{lrrrr}\noalign{\getlongtablewidth\global\LTcapwidth=\longtablewidth}
\caption{\label{tab:shiftcellmeans}Mixture model estimates for key transitions immediately preceding a sentence. Cell means are shown for transitions that do and do not involve the transition to the character following shift in msecs for the slowdown and the probability of hesitant transitions. The difference for including the transition duration to the character after shift is shown on log scale (for transition durations) and logit scale for probability of hesitant transitions. 95\% PIs in brackets.}\\
\toprule
Language & \multicolumn{1}{c}{\_\textasciicircum{}[shift] + C} & \multicolumn{1}{c}{\_\textasciicircum{}[shift]} & \multicolumn{1}{c}{Difference} & \multicolumn{1}{c}{BF}\\
\midrule
\endfirsthead
\caption*{\normalfont{Table \ref{tab:shiftcellmeans} continued}}\\
\toprule
Language & \multicolumn{1}{c}{\_\textasciicircum{}[shift] + C} & \multicolumn{1}{c}{\_\textasciicircum{}[shift]} & \multicolumn{1}{c}{Difference} & \multicolumn{1}{c}{BF}\\
\midrule
\endhead
Fluent transitions &  &  &  & \\
\ \ \ overall & 156 [145, 169] & 159 [147, 172] & -0.02 [-0.13, 0.09] & 0.06\\
Hesitation duration &  &  &  & \\
\ \ \ L1 & 1,167 [1,037, 1,305] & 1,281 [1,082, 1,496] & -0.06 [-0.21, 0.09] & 0.11\\
\ \ \ L2 & 2,019 [1,772, 2,292] & 1,802 [1,557, 2,073] & 0.12 [-0.02, 0.27] & 0.29\\
Hesitation probability &  &  &  & \\
\ \ \ L1 & 1.00 [.99, 1.00] & .78 [.69, .86] & 4.91 [3.68, 6.35] & > 100\\
\ \ \ L2 & 1.00 [.99, 1.00] & .92 [.87, .96] & 3.4 [2.12, 4.85] & > 100\\
\bottomrule
\addlinespace
\insertTableNotes
\end{longtable}

}

\end{ThreePartTable}
\end{center}

The fluent transition duration and hesitation duration were not affected
by whether or not the sentence-initial transition include the character
following shift for neither language. However, we found strong evidence
for an increased hesitation probability -- in both -- languages when the
before-sentence key transition included the character following the
shift key. Notably, the hesitation probability increased ceiling. It is
not entirely surprised that the constrained mixture model identifies
essentially all before-sentence transition as pauses: the constrained
model does not distinguish between transition locations for the duration
of fluent transitions. Therefore essentially all before-sentence
transitions are identified as hesitation as they always include two
keystrokes while all other transition location (which is the majority of
the data) do not.
\end{appendix}
